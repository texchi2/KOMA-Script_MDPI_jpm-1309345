%%%%%%%%%%%%%%%%%%%%%%%%%%%%%%%%%%%%%%%%%
% KOMA-Script Presentation
% LaTeX Template
% Version 1.1 (18/10/15)
%
% This template has been downloaded from:
% http://www.LaTeXTemplates.com
%
% Original Authors:
% Marius Hofert (marius.hofert@math.ethz.ch)
% Markus Kohm (komascript@gmx.info)
% Described in the PracTeX Journal, 2010, No. 2
%
% License:
% CC BY-NC-SA 3.0 (http://creativecommons.org/licenses/by-nc-sa/3.0/)
%
%%%%%%%%%%%%%%%%%%%%%%%%%%%%%%%%%%%%%%%%%

%----------------------------------------------------------------------------------------
%	PACKAGES AND OTHER DOCUMENT CONFIGURATIONS
%----------------------------------------------------------------------------------------
% KOMA-Script_MDPI_jpm-1309345
% compiling only on TexLive 2019

\documentclass[
paper=landscape,
paper=160mm:90mm, %128mm:96mm, % The same paper size as used in the beamer class
fontsize=11pt, % Font size
pagesize, % Write page size to dvi or pdf
parskip=half-, % Paragraphs separated by half a line
%captions=tableheading
]{scrartcl} % KOMA script (article)

\linespread{1.12} % Increase line spacing for readability



%~~~~~~~~~~~~~~~~~~~~~~~~~~~~~~~~~~~~~~~~~~~~~~~
% from main.tex
%%%%%%%%%%%%%%%%%%%%%%%%%%%%%%%%%% Tex
\maxdeadcycles=1000 % Output loop---200 consecutive dead cycles.

\usepackage{wrapfig} % for wrapfigure, text wrapping figure

\usepackage{todonotes}

\usepackage{floatrow} % for side caption (beside)
%\floatsetup[widefigure]{margins=hangleft,capposition=beside,capbesideposition={center,outside},floatwidth=\textwidth} % {center,outside}
\usepackage{amsmath,amsfonts}
\usepackage{tikz-imagelabels} % for  tikz
%\usepackage[]{caption} % bf
%\newcommand{\bcaption}[2]{\caption{\textbf{#1} #2}}
%\usepackage{outlines}
% mark in blue or red
%\usepackage{xcolor}

%\usepackage{xr-hyper}
\usepackage{hyperref}
%\newcommand{\R}[1]{\label{#1}\linelabel{#1}} % for \label page and line
%\newcommand{\R}[1]{\linelabel{#1}} % for \label page and line
%\newcommand{\lr}[1]{page~\pageref{#1} (line~\lineref{#1})} % for \ref page and line
% for xr in your preamble
% https://www.overleaf.com/learn/how-to/Cross_referencing_with_the_xr_package_in_Overleaf
%\makeatletter
%\newcommand*{\addFileDependency}[1]{% argument=file name and extension
%  \typeout{(#1)}
%  \@addtofilelist{#1}
%  \IfFileExists{#1}{}{\typeout{No file #1.}}
%}
%\makeatother

%\newcommand*{\myexternaldocument}[1]{%
%    \externaldocument{#1}%
%    \addFileDependency{#1.tex}%
%    \addFileDependency{#1.aux}%
%}
% my external document I would like to reference the labels of. 
%\myexternaldocument{Supplementary_figures} %.tex .aux

\usepackage{array}
\usepackage{ragged2e}
\usepackage{rotating}
\usepackage{tabularx} % for resizebox?
\usepackage{makecell}
% \usepackage{array}
\usepackage{multirow}
\usepackage{colortbl}
\usepackage{hhline}
\usepackage{siunitx} % for  1e-10 scientific notation
%\usepackage{caption}
%\usepackage{subcaption}
%\usepackage{subcaption} % for figure  side by side, subfigure

%DIF 140a141
\usepackage{pbox} %DIF > 
%DIF -------

\usepackage{booktabs, multirow} % for borders and merged ranges
\usepackage{soul}% for underlines
%\usepackage[table]{xcolor} % for cell colors
\usepackage{changepage,threeparttable} 

%%% for abbreviations, or acronyms
\usepackage[acronym, nopostdot]{glossaries}  % automake
\usepackage{glossary-inline}
%\setacronymstyle{long-short}
%\renewcommand*{\glossarysection}[2][]{} 
%\renewcommand*{\glossarysection}[2][]{\textbf{#1}: }
% for abbreviations environment
%\newcommand{\abbrlabel}[1]{\makebox[3cm][l]{\textbf{#1}\ \dotfill}}
\newenvironment{abbreviation}


%\section{abbreviation}
%%%%%%%%%%%%% Define abbreviation
%\makeglossaries %https://tex.stackexchange.com/questions/110095/list-of-acronyms-is-not-displayed



\newacronym{ncbi}{NCBI}{National Center for Biotechnology Information}
\newacronym{degs}{DEGs}{differentially expressed genes}

\newacronym{ihc}{IHC}{immunohistochemistry}
\newacronym{fdr}{FDR}{false discovery rate}

\newacronym{hpa}{HPA}{the Human Protein Atlas}
\newacronym{hnscc}{HNSCC}{head and neck squamous cell carcinoma}
\newacronym{tcga}{TCGA}{the Cancer Genome Atlas}
\newacronym{tcpa}{TCPA}{the Cancer Proteome Atlas}
\newacronym{rna}{RNA}{ribonucleic acid}
\newacronym{rnaseq}{RNA-Seq}{RNA sequencing}
\newacronym{lncrna}{lncRNA}{long non-coding RNA}
%\newacronym{km}{KM}{Kaplan--Meier}
\newacronym{rppa}{RPPAs}{reverse-phase protein arrays}
\newacronym{rpma}{RPMA}{reverse-phase protein lysate microarray}

\newacronym{mmp}{MMP}{matrix metalloproteinase}
 %DKK1, CAMK2N1, STC2, PGK1, SURF4, USP10, NDFIP1, FOXA2, STIP1, and DKC1
 %ZNF557, ZNF266, IL19, MYO1H, FCGBP, LOC148709, EVPLL, PNMA5, KIAA1683, and NPB

\newacronym{DKK1}{DKK1}{dickkopf WNT signaling pathway inhibitor 1} 
\newacronym{CAMK2N1}{CAMK2N1}{calcium/calmodulin dependent protein kinase II inhibitor 1} 
\newacronym{CALML5}{CALML5}{calmodulin like 5}

\newacronym{STC2}{STC2}{stanniocalcin 2} 
\newacronym{PGK1}{PGK1}{phosphoglycerate kinase 1} 
\newacronym{SURF4}{SURF4}{surfeit 4} 
\newacronym{USP10}{USP10}{ubiquitin specific peptidase 10} 
\newacronym{NEDD4}{NEDD4}{neural precursor cell expressed, developmentally down-regulated 4}
\newacronym{NDFIP1}{NDFIP1}{NEDD4 family interacting protein 1} 
\newacronym{FOXA2}{FOXA2}{forkhead box A2} 
\newacronym{STIP1}{STIP1}{stress-induced-phosphoprotein 1} 
\newacronym{DKC1}{DKC1}{dyskeratosis congenita 1, dyskerin} 

\newacronym{ZNF557}{ZNF557}{zinc finger protein 557} 
\newacronym{ZNF266}{ZNF266}{zinc finger protein 266} 
\newacronym{IL19}{IL19}{interleukin 19} 
\newacronym{MYO1H}{MYO1H}{myosin 1H} 
\newacronym{FCGBP}{FCGBP}{Fc fragment of IgG binding protein} 
\newacronym{LOC148709}{LOC148709}{LncRNA LOC148709} 
\newacronym{EVPLL}{EVPLL}{envoplakin-like protein} 
\newacronym{PNMA5}{PNMA5}{paraneoplastic antigen like 5} 
%\newacronym{KIAA1683}{KIAA1683}{IQCN, IQ Motif Containing N} 
\newacronym{IQCN}{IQCN}{IQ motif containing N} % previous name KIAA1683
% "IQ'' refers to the first two amino acids of the motif: isoleucine (commonly) and glutamine (invariably)
\newacronym{NPB}{NPB}{neuropeptide B} 

 \newacronym{rt}{RT}{radiation therapy}
 \newacronym{nccn}{NCCN}{National Comprehensive Cancer Network}
 \newacronym{hif}{HIF}{hypoxia-inducible factor}
 \newacronym{egfr}{EGFR}{epidermal growth factor receptor}
 \newacronym{ras}{RAS}{rat sarcoma}
 \newacronym{hras}{HRAS}{Harvey rat sarcoma viral oncoprotein}
 \newacronym{erk}{ERK}{extracellular signal-regulated kinases}
 \newacronym{us}{US}{United States}
 \newacronym{fda}{FDA}{Food and Drug Administration}
 \newacronym{tpf}{Tax-PF}{docetaxel, cisplatin, and 5-fluorouracil}
 \newacronym{tki}{TKI}{tyrosine kinase inhibitor}
 \newacronym{her}{HER}{human epidermal growth factor receptor}
 \newacronym{ici}{ICI}{immune-checkpoint inhibitor}
 \newacronym{ctla4}{CTLA-4}{cytotoxic T lymphocyte antigen 4}
 \newacronym{pd1}{PD-1}{programmed death 1}
 \newacronym{pdl1}{PD-L1}{programmed death ligand 1}
 \newacronym{tim3}{TIM-3}{T-cell immunoglobulin mucin protein 3}
 \newacronym{lag3}{LAG-3}{lymphocyte activation gene 3}
 \newacronym{ifng}{IFN-$\gamma$}{interferon gamma}
 \newacronym{tigit}{TIGIT}{T cell immunoglobin and immunoreceptor tyrosine-based inhibitory motif}
 \newacronym{gitr}{GITR}{glucocorticoid-induced tumor necrosis factor receptor}
 \newacronym{vista}{VISTA}{V-domain Ig suppressor of T-cell activation}
 \newacronym{tmsb4x}{TMSB4X}{thymosin beta-4 X-linked}
 \newacronym{emt}{EMT}{epithelial-mesenchymal-transition}
 \newacronym{gdc}{GDC}{Genomic Data Commons}
 \newacronym{nci}{NCI}{the National Cancer Institute}
 \newacronym{gdac}{GDAC}{genome data analysis center}
 \newacronym{rest}{REST}{Representational State Transfer} 
 \newacronym{api}{API}{application programmable interface}
\newacronym{grch38}{GRCh38}{Genome Reference Consortium Homo sapiens genome assembly 38}
\newacronym{fpkm}{FPKM}{Fragments per kilobase per million reads mapped}
\newacronym{rsem}{RSEM}{RNA-Seq by Expectation-Maximization}
\newacronym{slca}{SLC35E2A}{solute carrier family 35 member E2A}
\newacronym{slcb}{SLC35E2B}{solute carrier family 35 member E2B}
\newacronym{cde}{CDE}{Common Data Element}
\newacronym{id}{ID}{identification}
\newacronym{ajcc}{AJCC}{the American Joint Committee on Cancer}
\newacronym{uicc}{UICC}{he Union for International Cancer Control}
\newacronym{tnm}{TNM}{the tumor size (T), cervical lymph node metastases (N), and distal metastasis status (M)}
\newacronym{ci95}{95\% CI}{95\% confidence interval}
\newacronym{os}{OS}{overall survival}
\newacronym{hr}{HR}{hazard ratio}
\newacronym{hpv}{HPV}{human papillomavirus}
\newacronym{ene}{ENE}{extra-nodal extension}
\newacronym{lvsi}{LVSI}{lymph-vascular space invasion}
\newacronym{pni}{PNI}{perineural invasion}
\newacronym{doi}{DOI}{depth of invasion}
\newacronym{lnd}{LND}{lymph node density}
\newacronym{wpoi5}{WPOI-5}{worst pattern of invasion score 5}
\newacronym{glut4}{GLUT4}{glucose transporter 4}
\newacronym{slc2a4}{SLC2A4}{solute carrier family 2 member A4}
\newacronym{trim24}{TRIM24}{tripartite motif-containing 24}
\newacronym{til}{TIL}{tumor-infiltrating lymphocytes}
\newacronym{tmb}{TMB}{tumor mutational burden}




%------------------------------------------------
\usepackage{outlines}
\usepackage[font={small}]{caption}
\newcommand{\bcaption}[2]{\caption{\textbf{#1} #2}}


\usepackage[margincaption]{sidecap}

\usepackage{subcaption} % for figure  side by side, subfigure

%%%%%% updated 2020 by esdd
%% https://tex.stackexchange.com/questions/547723/latex-throws-errors-on-section
%------------------------------------------------
% Colors
\usepackage[]{xcolor}  % table % Required for custom colors
% Define a few colors for making text stand out within the presentation
\definecolor{mygreen}{RGB}{44,85,17}
\definecolor{myblue}{RGB}{34,31,217}
\definecolor{mybrown}{RGB}{194,164,113}
\definecolor{myred}{RGB}{255,66,56}
% Use these colors within the presentation by enclosing text in the commands below

\newcommand*{\mygreen}[1]{\textcolor{mygreen}{#1}}
\newcommand*{\myblue}[1]{\textcolor{myblue}{#1}}
\newcommand*{\mybrown}[1]{\textcolor{mybrown}{#1}}
\newcommand*{\myred}[1]{\textcolor{myred}{#1}}

\definecolor{asparagus}{rgb}{0.53, 0.66, 0.42}

\newenvironment{MyColorPar}[1]{% for RED marked of revision
    \leavevmode\color{#1}\ignorespaces%
}{%
}%


% Margins
\usepackage[ % Page margins settings
  includeheadfoot,
  top=-1mm,
  bottom=1.5mm,
  left=1mm,
  right=3.5mm,
  headsep=0mm,
  footskip=8.5mm
]{geometry}

% Fonts
\usepackage[T1]{fontenc}     % For correct hyphenation and T1 encoding
\usepackage{lmodern} % Default font: latin modern font
%\usepackage{fourier} % Alternative font: utopia
%\usepackage{charter} % Alternative font: low-resolution roman font
\renewcommand{\familydefault}{\sfdefault} % Sans serif - this may need to be commented to see the alternative fonts

% Various required packages
\usepackage{amsthm} % Required for theorem environments
\usepackage{bm} % Required for bold math symbols (used in the footer of the slides)
\usepackage{tikz} % Required for colored boxes, loads graphicx and other packages
\usepackage{booktabs} % Required for horizontal rules in tables
\usepackage{multicol} % Required for creating multiple columns in slides
\usepackage{lastpage} % For printing the total number of pages at the bottom of each slide

\usepackage{microtype} % Better typography

% Slide layout configuration
\usepackage[automark]{scrlayer-scrpage} % Required for customization of the header and footer
\clearpairofpagestyles % Remove the default header and footer
\AddLayersAtBeginOfPageStyle{scrheadings}{headerbg,footerbg}% add background layers
\setkomafont{pageheadfoot}{\normalfont\sffamily} % Font settings for the header and footer
\setkomafont{pagehead}{\color{white}}
% Header configuration - if you don't want a header remove this block
\DeclareNewLayer[
  background,
  head,
  hoffset=0pt,
  width=\paperwidth,
  mode=picture,
  contents=\putLL{\myblue{\rule{\layerwidth}{\layerheight}}}
]{headerbg}

\ihead{\rightbotmark}

% Footer configuration
\KOMAoptions{footwidth=\textwidth+2mm:0pt}
\setkomafont{pagefoot}{\color{black}\tiny} % Small font size for the footnote
%\DeclareNewLayer[
  %background,
  %foot,
  %hoffset=0pt,
  %width=\paperwidth,
  %mode=picture,
  %contents=\putLL{\myblue{\rule{\layerwidth}{\layerheight}}}
%]{footerbg}
\ifoot{\myauthor\ \raisebox{0.2mm}{$\bm{\vert}$}\ \myuni} % Left side text
\ofoot*{\pagemark/\pageref{LastPage}} % Right side

% Sets vertical centering of slide contents with increased space between paragraphs/lists
\makeatletter
\renewcommand*{\@textbottom}{\vskip \z@ \@plus 1fil}
\newcommand*{\@texttop}{\vskip \z@ \@plus .5fil}
\setparsizes{1em}{\z@\@plus .25fil}{0pt plus 1fil}
\makeatother

% Remove page numbers and the dots leading to them from the outline slide
\DeclareTOCStyleEntries[linefill=\hfill,pagenumberbox=\gobble]{section}{section,subsection,subsubsection}
%\newcommand*\gobble[1]{}
\AfterTOCHead{\small}

%\renewcaptionname{english}{\contentsname}{Outline} % contentsname % Change the name of the table of contents
% Section spacing - deeper section titles are given less space due to lesser importance
\RedeclareSectionCommands[beforeskip=0pt,afterskip=0pt,afterindent=true,runin=false]{section,subsection,subsubsection}
\RedeclareSectionCommand[afterskip=-1mm]{subsection}
\RedeclareSectionCommand[afterskip=-2mm]{subsubsection}
\setcounter{secnumdepth}{\partnumdepth} % How deep sections are numbered

% Theorem style
\newtheoremstyle{mythmstyle} % Defines a new theorem style used in this template
  {0.5em} % Space above
  {0.5em} % Space below
  {} % Body font
  {} % Indent amount
  {\sffamily\bfseries} % Head font
  {} % Punctuation after head
  {\newline} % Space after head
  {\thmname{#1}\ \thmnote{(#3)}} % Head spec

\theoremstyle{mythmstyle} % Change the default style of the theorem to the one defined above
\newtheorem{theorem}{Theorem}[section] % Label for theorems
\newtheorem{remark}[theorem]{Remark} % Label for remarks
\newtheorem{algorithm}[theorem]{Algorithm} % Label for algorithms


% The code for the box which can be used to highlight an element of a slide (such as a theorem)
\newcommand*{\mybox}[2]{% The box takes two arguments: width and content
  \par\noindent
  \begin{tikzpicture}[mynodestyle/.style={rectangle,draw=myblue,thick,inner sep=2mm,text justified,top color=white,bottom color=white,above}]
    \node[mynodestyle,at={(0.5*#1+2mm+0.4pt,0)}]{% Box formatting
      \begin{minipage}[t]{#1}
      #2
      \end{minipage}%
    };
  \end{tikzpicture}
\par\vspace{-1.3em}}

%----------------------------------------------------------------------------------------
%	PRESENTER INFORMATION
%----------------------------------------------------------------------------------------
% 
\newcommand*{\mytitle}{A Transcriptomic Analysis of Head and Neck Squamous Cell Carcinomas for Prognostic Indications} % old Title Transcriptomic Analysis for Prognostic Value in Head and Neck Squamous Cell Carcinoma
\newcommand*{\runninghead}{HNSCC Biomarker} % Running head displayed on almost all slides
\newcommand*{\myauthor}{Li-Hsing (Tex) Chi} % Presenters name(s)
\newcommand*{\myadvisorM}{Michael Hsiao}
\newcommand*{\myadvisorJ}{Yu-Chuan (Jack) Li}
\newcommand*{\mydate}{2021/09/10} %{\today} % Presentation date
\newcommand*{\myuni}{Taipei Medical University} % University or department

%----------------------------------------------------------------------------------------

\begin{document}

%----------------------------------------------------------------------------------------
%	TITLE SLIDE
%----------------------------------------------------------------------------------------

% Title slide - you may have to tweak a few of the numbers if you wish to make changes to the layout
\thispagestyle{empty} % No slide header and footer
\begin{tikzpicture}[remember picture,overlay] % Background box
  \node [xshift=\paperwidth/2,yshift=\paperheight/2] at (current page.south west)
    [rectangle,fill,inner sep=0pt,minimum width=\paperwidth,minimum height=\paperheight/1.7,top color=myblue,bottom color=myblue]{}; % Change the height of the box, its colors and position on the page here
\end{tikzpicture}
% Text within the box
\begin{flushright}
  \vspace{1.6cm}
  \color{white}\sffamily
  {\bfseries\Large\mytitle\par}% Title
  \vspace{0.5cm}
  \normalsize
  Speaker: \myauthor  \hspace{28.7mm} \par % Author name
  Advisors: \myadvisorM \hspace{3mm} \myadvisorJ\par
  \mydate\par % Date
  \myuni\par
  \vfill
\end{flushright}
%Chi, L.-H., Wu, A. T. H., Hsiao, M. \& Li, Y.-C. (Jack). Journal of Personalized Medicine, 11, 782 (2021).
%\cite{Chi2021}

\clearpage

\begin{figure}
    \centering
    \includegraphics[width=16cm]{Article_Banner_MDPI_jpm-11-00782.pdf}
%    \caption{Caption}
%    \label{fig:my_label}
\end{figure}
\clearpage
%----------------------------------------------------------------------------------------
%	TABLE OF CONTENTS
%----------------------------------------------------------------------------------------

\thispagestyle{empty} % No slide header and footer

\small\tableofcontents % Change the font size and print the table of contents - it may be useful to shrink the font size further if the presentation is full of sections
% To exclude sections/subsections from the table of contents, put an asterisk after \(sub)section like so: \section*{Section Name}

\clearpage

%----------------------------------------------------------------------------------------
%	PRESENTATION SLIDES
%----------------------------------------------------------------------------------------

%\section*{Vize}

%\clearpage

%------------------------------------------------




\section{Introduction}

%\begin{figure}
%\end{figure}



\begin{minipage}[c]{0.30\linewidth}
\begin{outline}
    \1 Survival analysis of the Cancer Genome Atlas (TCGA) dataset for HNSCC
        \2 gene expression-based prognostic biomarkers
        \2 a \hl{cutoff} point is usually used in survival analysis
\end{outline}
\end{minipage}%\hspace{2mm}
%\begin{wrapfigure}{r}{0.3\textwidth}
\begin{minipage}[c]{0.65\linewidth}
    \raggedright
    \hfill\includegraphics[width=8cm]{Figure_1_manuscript_workflow_modify.pdf}
\end{minipage}




\clearpage

\begin{figure}
  \begin{minipage}[c]{0.35\linewidth}
  \begin{picture}(15, 20) % size %(1,0.55038404)%
\centering
  \put(5,-65){\includegraphics[width=7.0cm]{graphic_abstract_pvalueTex.pdf}}% X-axis: hyphen to minus sign
%    \includegraphics[height=7.0cm, width=7.5cm]{graphic_abstract_pvalueTex.pdf}%
    \captionsetup{labelformat=empty}
%    \caption{Figure. Mindfulness meditation in \hl{holistic} cancer care}
    
  \put(128.5, 78.5){%\fontfamily{qcr}\selectfont
  \large (1)}
  \put(178.5, 48.5){%\fontfamily{qcr}\selectfont
  \large (2)}
  \put(188.5, 10){%\fontfamily{qcr}\selectfont
  \large (3)}
  \put(178.5, -10){%\fontfamily{qcr}\selectfont
  \large (4)}
  \put(118.5, -40){%\fontfamily{qcr}\selectfont
  \large (5)}
  \put(2,-105){Figure. Mindfulness meditation in \hl{holistic} cancer care}
  
  \end{picture}
  \end{minipage}\hfill
  \begin{minipage}[c]{0.55\linewidth}
    \centering
    \begin{outline}[enumerate]
    \0 An in-house workflow---pvalueTex---in Rstudio server
        \1 data retrieving from TCGA via FirebrowseR package
        \1 pre-processing
            \2 data cleaning
            \2 feature selection
        \1 \hl{sliding-window} cutoff mining engine for Kaplan--Meier survival analysis to find a optimal cutoff
        \1 Cox proportional hazard modeling
        \1 validation by using the another independent HNSCC dataset (GSE65858)~\cite{Wichmann2015}

    \end{outline}
%    \captionof{table} % by KOMA-script
%      {%
%        Different kinds of ducks%
%        \label{tab:duck}%
 %     }
  \end{minipage}
\end{figure}

\clearpage


Keyword:\\
head and neck squamous cell carcinoma (HNSCC);\\
the Cancer Genome Atlas (TCGA); \\
transcriptomic analysis; survival analysis; \\
\hl{optimal cutoff} for Kaplan--Meier curve; \\
effect size; hazard ratio in Cox's modelling;\\
\acrfull{CAMK2N1}; \\
\acrfull{CALML5}; \\
\acrfull{FCGBP}; \\
\hl{mindfulness meditation}



\clearpage

\thispagestyle{headings}
\markright{Introduction\hfill Why TCGA dataset? \hfill}

The advantages of applying the TCGA data for cancer biomarker identification include:
\begin{outline}
\1  the TCGA database has the \hl{largest collection} (cancer types and cohort size), especially in HNSCC)
%There are many physical and social features of patients available for survival modeling. % ($X_1 ... X_n$)
\2 the whole-genome sequencing data were harmonized
\2 the essential demographic data, physical and social features of patients (exposure to alcohol, asbestos, radioactive radon, \hl{tobacco} smoking, and cigarettes)
\2 computational and life scientists who study cancer designed useful web-based tools and $APIs^{*}$
\1 getting \hl{help} soon from the research community for trouble-shooting purposes
\1 many achievements and getting published, including \hl{Chi} et al., 2021~\cite{Chi2021}.
    \2 "an immeasurable source of knowledge"~\cite{Tomczak2015}
    \2 "The results published~\cite{Chi2021} and shown here are in part based upon data generated by the TCGA Research Network: \url{https://www.cancer.gov/tcga}."
\end{outline}

\tiny * Application Programming Interface (program talks to program)

\clearpage

%%
\section{Materials and Methods} % 4

\large Data obtained through the Broad Firebrowse $\acrshort{rest}ful^{*}$ \acrshort{api} with an R software package, \hl{FirebrowseR} (available at \url{https://github.com/mariodeng/FirebrowseR})~\cite{Deng2017}

\subsection{Two Patient Cohorts} 
\large TCGA's \acrfull{hnscc}: n = 414, for \hl{modelling}\\ %528\\
\indent{GSE65858 cohorty~\cite{Wichmann2015}: n = 270, for \hl{validation}}%288}

\subsubsection{RNA Sequencing Data} 
20,500 \hl{RNA-Seq} data of protein-coding genes\\%, dierived from cancer specimen of each participant


\subsubsection{The Clinicopathological Data}
gender, age, clinical tumor size (T), clinical cervical lymph node metastases (N), clinical distant metastasis (M), \hl{pathological surgical margin}, tobacco exposure, and \hl{survival data}.

\tiny * \acrfull{rest}

\clearpage
\subsubsection{Demographic data}
% a table
% Please add the following required packages to your document preamble:
% \usepackage{graphicx}
\begin{table}[H]
\centering
%\caption{}
%\label{tab:my-table}
\resizebox{0.7\textwidth}{!}{%
\begin{tabular}{|l|l|l|}
\hline
Features         & TCGA HNSCC   & GSE65858    \\ \hline
Participants    & 414   & 270\\ \hline
Gender (females) & 26.9\% & 17.0\% \\ \hline
Median age       & 61 y/o       & 58 y/o      \\ \hline
Stage I/II       & 20.7\%          & 19.0\%      \\ \hline
Smokers          & 97.5\%       & 82.2\%      \\ \hline
Alcohol          & 67.6\%       & 88.5\%      \\ \hline
HPV(-)           & 82.1\%       & 84.9\%      \\ \hline
\end{tabular}%
}
\end{table}

\clearpage




%%%
\thispagestyle{headings}
\markright{Workflow\hfill Our workflow---pvalueTex \hfill}
\begin{figure}

  \begin{minipage}[c]{0.45\linewidth}

The advantages of our workflow---pvalueTex:
\begin{outline}
\1 A model for biomarker estimates %(Figure \ref{fig:figure1})
    \2 scanning 20,500 human protein-coding genes
\1 The Purpose of Sliding-Window Cutoff Selection
    \2 to find an \hl{optimal cutpoint} of that \acrshort{rna} expression data 
    \2 to maximize candidate mining coverage---\hl{catching more}
%    \2 identify more but sometimes weak "biomarkers"
    \2 validation by the other cohort---\hl{stringent checking}
\end{outline}
\end{minipage}
% figure replaced by minipage
%\clearfloatsetup{figure}
%\floatsetup[figure]{style=no,capposition=beside,capbesideposition={center,outside},capbesideframe=yes,facing=yes}
\hfill
%\todo{python code for all figures}
\begin{minipage}[c]{0.5\linewidth}
%\label{fig:figure1}
%\floatbox[{\capbeside\thisfloatsetup{capbesideposition={right,center},capbesidewidth=.35\linewidth,capbesidesep=quad}}]{figure}[\FBwidth]
%\centering
%\widefigure, keepaspectratio
\includegraphics[width=8cm]{Figure_1_manuscript_workflow.pdf} % .PDF is better than .png
%, height=8cm
%\caption % , step 1 (\textcolor{blue}{blue line}: main procedure) and step 2 (\textcolor{orange}{orange line}: analysis export).
%Step 3 (purple line: dealing with surgical margin).
%\sidecaptionvpos{c}
%{\caption{\hl{A workflow of \acrshort{hnscc} biomarker discovery.}
%The workflow includes data retrieval from the TCGA GDC data portal, data processing with merging and cleaning, and then performing the survival analyses (within \textcolor{yellow}{yellow} square).}} %The Cutoff engine (in R script: cutofFinder\_func.HNSCC.R, a serial cutoff for grouping patients with \textcolor{asparagus}{low} or \textcolor{red}{high} expression of a specific gene, to yield a collection of \protect\textit{P} values; please see Materials and Methods section for details) might calculate all possible Kaplan--Meier \protect\textit{P} values (corrected by \acrlong{fdr}, FDR, method) to find the optimal cutoff value of gene expression for subsequent Cox modeling. The candidate selection performs (1) dissecting and selection of candidate genes with further Bonferroni adjusted \protect\textit{P} values and the hazard ratios of a Cox model, based on the results from the survival analyses; (2) survival analyses of the other HNSCC dataset (GSE65858) using Kaplan--Meier estimates (with FDR corrections) and Cox modeling.\\ The biomarker candidates were consensus results of TCGA and GSE65858. (HNSCC: head and neck squamous cell carcinoma; TCGA: the Cancer Genome Atlas; RNA-Seq: RNA sequencing; GDC: Genomic Data Commons.)} } % end of caption

% Description:1) FDR correction of Kaplan--Meier \protect\textit{P} values during Cutoff finding; and 2) Bonferroni correction of Kaplan--Meier \protect\textit{P} values after Cox modeling for candidate selection.

\end{minipage}
\end{figure}

\clearpage
%%%%%%%%%%%%%%%%%
% automatic generation by jpm2KOMO-script.py
%% [2011/08/08]
%%%%%%%

%\section{Materials and Methods}

\thispagestyle{headings}
\markright{Statistics}%\hfill Why TCGA dataset? \hfill}

\subsection{Cutoff Finder Core Engine}
%\subsection{Statistical Consideration for Survival Analysis}
\begin{minipage}[c]{0.45\linewidth}


\large
\begin{outline}
\1 sliding-window cutoff selection
    \2 a serial cut (30th -- 70th percentiles of the cohort)
    \2 log-rank test for Kaplan--Meier estimates
    \2 the lowest \textit{p} values in plot
\end{outline}
\end{minipage}
\begin{minipage}[c]{0.35\linewidth}
\includegraphics[width=8cm]{Rplot_pvaluePlot_NDFIP1.pdf}
\end{minipage}



\clearpage
\subsection{Biomarker Selection and Validation}



\clearpage



\section{Results}

\thispagestyle{myheadings}
\markright{Results\hfill \textit{p} value plots \hfill}


\begin{figure}[ht]
%\subsubsection{Figure 2:h}
%\begin{figure}[H]
%    \captionsetup[subfigure]{
%  font=footnotesize,
%  justification=raggedright
%  skip=10pt
%    }
%    \setlength{\abovecaptionskip}{35pt plus 3pt minus 2pt} % Chosen fairly arbitrarily
%\centering
%\floatbox[{\capbeside\thisfloatsetup{capbesideposition={right,center},capbesidewidth=.35\linewidth,capbesidesep=quad}}]{figure}[\FBwidth]
%\centering
% not \widefigure
%\includegraphics[width=14cm]{Figure2.pdf} %MDPI: Please revise the scientific notation format to a × 10b, and picture is not clear.ok
%\https://en.wikibooks.org/wiki/LaTeX/Floats,_Figures_and_Captions
% 2 by 2


%% ab
    \begin{subfigure}[t]{0.45\textwidth}
%\subfloat[Subfigure 1 list of figures text][(a)]{
%        \vspace*{-5mm}
%        \caption{~}
%        \fbox{
        \includegraphics[width=6.5cm]{Rplot02_FDRP_uniHR.pdf}  %}    %{Rplot02_rawP_uniHR.pdf}

    \end{subfigure} \hfill
%\subfloat[Subfigure 1 list of figures text][(b)]{
    \begin{subfigure}[t]{0.45\textwidth}
%        \caption{}
        \includegraphics[width=6.5cm]{Rplot02_FDRP_multiHR.pdf}
%        \caption{}
    \end{subfigure} \\
%\end{adjustwidth}
%    \vskip\baselineskip
% ,clip,keepaspectratio, ,height=6.5cm
%\vspace{0.3cm}

\caption{
%The effect size was estimated by Cox's hazard ratio.
Step 1: selection by Kaplan--Meier (KM) survival \textit{p}~value, 
(\textbf{a}) univariate HR versus FDR-adjusted \textit{p}~value; (\textbf{b}) multivariate HR versus FDR-adjusted \textit{p}~value.}

\end{figure}

\clearpage
\begin{figure}
%% cd
%\begin{adjustwidth}{-3em}{0em}
%\subfloat[Subfigure 1 list of figures text][(c)]{
    \begin{subfigure}[b]{0.45\textwidth}
%        \caption{}
        \includegraphics[width=6.5cm]{Rplot02_BonferroniP_uniHR.pdf}
%        \caption{}
    \end{subfigure} \hfill
%\qquad
%\subfloat[Subfigure 1 list of figures text][(d)]{
    \begin{subfigure}[b]{0.45\textwidth}
%        \caption{}
        \includegraphics[width=6.5cm]{Rplot02_BonferroniP_multiHR.pdf}
%        \caption{}
    \end{subfigure}
 % end of {\includegraphics}

\caption{
%After stringent restriction by Bonferroni-adjusted \textit{p}~values and Cox's HR, a few top-ranked genes were acquired by
Step 2: filtering by Bonferroni corrected KM \textit{p}~value, 
(\textbf{c}) univariate HR versus \textit{p}~value; (\textbf{d}) multivariate HR versus \textit{p}~value.}
%(TCGA: \acrlong{tcga}; HR: hazard ratio; FDR: \acrlong{fdr}).}

%\label{fig:figure2}
\end{figure}% * = no overlapping with text


%%

\clearpage
\begin{figure}[ht]

\floatbox[{\capbeside\thisfloatsetup{capbesideposition={right,center},capbesidewidth=.35\linewidth,capbesidesep=quad}}]{figure}[\FBwidth]
    %\centering
{    \includegraphics[width=9cm]{Rplot_TCGA_HNSCC_CoxHR_CAMK2N1_top3FDRKM.pdf}}
{    \caption{A volcano plot of 20 candidate genes in TCGA HNSCC.
%X axis: unadjusted \protect\textit{p}~value of Kaplan--Meier survival (-log10 transformed).
%Y axis: multivariate hazard ratio from Cox proportional regression model.
%Dotted line: significant Bonferroni corrected \protect\textit{p}~value. 
%\textcolor{red}{Red dots} mark 10 genes (unvalidated), which impact on poor prognosis ($HR>=1.5$). \textcolor{green}{Green dots} mark 10 genes (unvalidated), which affect on better survival ($HR<=0.5$).
%    This cohort was applied for exploration of the candidate biomarkers.
%    A total of 9416 genes had %\acrshort{fdr}-
%    unadjusted \protect\textit{p}~values of less than 0.05.
    \textcolor{red}{CAMK2N1}, \textcolor{red}{CALML5}, \textcolor{red}{FCGBP}, and 17 other genes (marked in \textcolor{black}{black square}) had hazard ratios (HRs) >$1.8$ or <$0.6$.
%    The 22 genes, listed on the side, had hazard ratios between 0.6 and 1.5.
    \textcolor{red}{Red spots}: $HR$ > 1.0.
    \textcolor{green}{Green spots}: $HR$ < 1.0.
%    (X-axis: Kaplan--Meier survival estimates, with \acrshort{fdr}-adjusted \protect\textit{p}~values (log10 transformed);
%y-axis: HR of Cox proportional hazard regression model.)
    }}
%%%\label{fig:hazards3}
\end{figure}

\clearpage
\begin{figure}[ht]

\floatbox[{\capbeside\thisfloatsetup{capbesideposition={right,center},capbesidewidth=.35\linewidth,capbesidesep=quad}}]{figure}[\FBwidth]
   % \centering
{    \includegraphics[width=9cm]{Rplot_GSE65858_CoxHR_CAMK2N1_top3FDRKM.pdf}}
{    \caption{Volcano plot of genes in survival analyses of a validation (GSE65858) cohort.
    % GSE117973 using the same platform GPL10558
The candidate genes---\textcolor{red}{CAMK2N1}, \textcolor{red}{CALML5}, and \textcolor{red}{FCGBP}---was confirmed.
%    In total, 534 genes had \acrshort{fdr}-adjusted \protect\textit{p}~values less than 0.05
    \textcolor{red}{Red spots}: hazard ratios are greater than 1.0;
    \textcolor{green}{Green spots}: hazard ratios are under 1.0.
%    The 22 genes, listed on the side, had hazard ratios >$1.8$ or <$0.6$.
%    (X-axis: Kaplan--Meier survival estimates, with \acrshort{fdr}-adjusted \protect\textit{p}~values, log10 transformed; y-axis: the hazard ratio (HR) under the Cox proportional hazard regression model).
    }}
%%%\label{fig:hazards534}
\end{figure}


\clearpage
\begin{figure}[ht]

\floatbox[{\capbeside\thisfloatsetup{capbesideposition={right,center},capbesidewidth=.45\linewidth,capbesidesep=quad}}]{figure}[\FBwidth]
% TCGA FDR Pvalue of CAMK2N1 (1.628308e-05), IL19 ( 6.543871e-06), FCGBP (4.827833e-05), CALML5 (0.0001970348)
{\setlength{\unitlength}{.78cm}
\begin{picture}(9, 10) %(1,0.55038404)%
\centering
  \put(0,0){\includegraphics[height=7.5cm]{Figure_4_CAMK2N1_CALML5_FCGBP.pdf}}%
  \put(1.3, 7.25){\fontfamily{qcr}\selectfont
  \tiny *\protect\textit{p} = \num{1.63e-05}}%CAMK2N1
    \put(1.3, 4.25){\fontfamily{qcr}\selectfont
  \tiny *\protect\textit{p} = \num{1.97e-4}}%IL19 6.54e-06 -> CALML5 0.0001970348
    \put(1.3, 1.15){\fontfamily{qcr}\selectfont
  \tiny *\protect\textit{p} = \num{4.83e-05}}%FCGBP

%\begin{annotationimage}{width=15cm}{Figure_4_CAMK2N1_IL19_FCGBP.pdf}
%\includegraphics[width=15cm]{Figure_4_CAMK2N1_IL19.pdf}
%\draw[annotation left = {Aries at 0.3}]

%\end{annotationimage}
\end{picture}%
}
%\hfill
{\caption{Kaplan--Meier survival analyses, during cutoff finding.
The Kaplan--Meier curves of (\textbf{a}) CAMK2N1, (\textbf{c}) CALML5, and (\textbf{e}) FCGBP with optimal \protect\textit{p}~values. 
%((b) the cutoffs derived from it's cumulative \protect\textit{p}~value plot).
%(c) Kaplan--Meier plot of IL19 under optimal \protect\textit{p}~value;
%((d) the cutoffs derived from it's cumulative \protect\textit{p}~value plot).
%(e) Kaplan--Meier plot of FCGBP under optimal \protect\textit{p}~value.\\
%The optimal cutoff value for CAMK2N1 is 
The cutoffs in the cumulative FDR-adjusted \protect\textit{p}~value plots of (\textbf{b}) CAMK2N1, (\textbf{d}) CALML5, and (\textbf{f}) FCGBP.%,
%show that over 50\% of those unadjusted \protect\textit{p}~values derived by the sliding-window cutoff-finding procedure are below 0.001.
(* \protect\textit{p}: \protect\textit{p}~value adjusted by \acrlong{fdr}, \acrshort{fdr}.)
}}
%%%\label{fig:figure4}
\end{figure}


%% table 1
\clearpage
\begin{table}[H] 
%\centering
\captionabove{Univariate/multivariate Cox proportional hazard regression analyses on overall survival time of CAMK2N1 gene expression in HNSCC.}

%%%\label{table:table2}
\arrayrulecolor[rgb]{0.255,0.255,0.255}


\resizebox{0.55\linewidth}{!}{%
\begin{tabular}{|l|l|c|c|c|c|c|c|} 
\noalign{\hrule height 1.0pt}
%\begin{tabularx}{\textwidth}{|p{2.5cm}|l|l|l|l|l|l|l|} 
%\arrayrulecolor{black}\cline{1-2}
\arrayrulecolor[rgb]{0.255,0.255,0.255}
\cline{3-8}
\multicolumn{2}{|l!{\color{black}\vrule}}{\multirow{2}{*}{\textbf{Features}}}                                                          & \multicolumn{3}{c|}{\textbf{Univariate}}                                                                                                                                                                                                                & \multicolumn{3}{c|}{\textbf{Multivariate}}                                                                                                                                                                                                               \\ 
\cline{3-8}
\multicolumn{2}{|l!{\color{black}\vrule}}{}                                                                                   & \multicolumn{1}{l!{\color{black}\vrule}}{\textbf{HR}}                                   & \multicolumn{1}{c!{\color{black}\vrule}}{\textbf{CI95\%}}                              & \multicolumn{1}{l!{\color{black}\vrule}}{\textbf{\protect\textit{p}~Value}}                    & \multicolumn{1}{l!{\color{black}\vrule}}{\textbf{HR}}                                   & \multicolumn{1}{c!{\color{black}\vrule}}{\textbf{CI95\%}}                              & \multicolumn{1}{l!{\color{black}\vrule}}{\textbf{\protect\textit{p}~Value}}                     \\ 
\arrayrulecolor{black}\hline
\multirow{2}{*}{Gender}                 & \multicolumn{1}{l!{\color{black}\vrule}}{{\cellcolor[rgb]{0.62,0.812,0.878}}Female} & \multicolumn{1}{l!{\color{black}\vrule}}{{\cellcolor[rgb]{0.62,0.812,0.878}}1} & \multicolumn{1}{l!{\color{black}\vrule}}{{\cellcolor[rgb]{0.62,0.812,0.878}}} & \multicolumn{1}{l!{\color{black}\vrule}}{{\cellcolor[rgb]{0.62,0.812,0.878}}} & \multicolumn{1}{l!{\color{black}\vrule}}{{\cellcolor[rgb]{0.62,0.812,0.878}}1} & \multicolumn{1}{l!{\color{black}\vrule}}{{\cellcolor[rgb]{0.62,0.812,0.878}}} & \multicolumn{1}{l!{\color{black}\vrule}}{{\cellcolor[rgb]{0.62,0.812,0.878}}}  \\ 
\cline{2-8}
                                        & Male                                                                                & 1.157                                                                          & 0.843--1.587                                                                   & 0.367                                                                         & 1.076                                                                          & 0.767--1.510                                                                   & 0.671                                                                          \\ 
\arrayrulecolor[rgb]{0.255,0.255,0.255}\hline
\multirow{2}{*}{Age at diagnosis}       & {\cellcolor[rgb]{0.62,0.812,0.878}}$<=65y$                                             & {\cellcolor[rgb]{0.62,0.812,0.878}}1                                           & {\cellcolor[rgb]{0.62,0.812,0.878}}                                           & {\cellcolor[rgb]{0.62,0.812,0.878}}                                           & {\cellcolor[rgb]{0.62,0.812,0.878}}1                                           & {\cellcolor[rgb]{0.62,0.812,0.878}}                                           & {\cellcolor[rgb]{0.62,0.812,0.878}}                                            \\ 
\cline{2-8}
                                        & $>65y$                                                                                 & 1.329                                                                          & 0.990--1.784                                                                   & 0.058                                                                         & 1.391                                                                          & 1.025--1.888                                                                   & \textcolor{red}{0.034}                                                         \\ 
\hline
\multirow{2}{*}{Clinical T Status}      & {\cellcolor[rgb]{0.62,0.812,0.878}}T1+T2                                            & {\cellcolor[rgb]{0.62,0.812,0.878}}1                                           & {\cellcolor[rgb]{0.62,0.812,0.878}}                                           & {\cellcolor[rgb]{0.62,0.812,0.878}}                                           & {\cellcolor[rgb]{0.62,0.812,0.878}}1                                           & {\cellcolor[rgb]{0.62,0.812,0.878}}                                           & {\cellcolor[rgb]{0.62,0.812,0.878}}                                            \\ 
\cline{2-8}
                                        & T3+T4                                                                               & 1.409                                                                          & 1.028--1.931                                                                   & \textcolor{red}{0.033}                                                        & 1.982                                                                          & 1.048--3.745                                                                   & \textcolor{red}{0.035}                                                         \\ 
\hline
\multirow{2}{*}{Clinical N Status}      & {\cellcolor[rgb]{0.62,0.812,0.878}}N0                                               & {\cellcolor[rgb]{0.62,0.812,0.878}}1                                           & {\cellcolor[rgb]{0.62,0.812,0.878}}                                           & {\cellcolor[rgb]{0.62,0.812,0.878}}                                           & {\cellcolor[rgb]{0.62,0.812,0.878}}1                                           & {\cellcolor[rgb]{0.62,0.812,0.878}}                                           & {\cellcolor[rgb]{0.62,0.812,0.878}}                                            \\ 
\cline{2-8}
                                        & N1-3                                                                                & 1.185                                                                          & 0.890--1.577                                                                   & 0.246                                                                         & 1.145                                                                          & 0.801--1.636                                                                   & 0.457                                                                          \\ 
\hline
\multirow{2}{*}{Clinical M Status}      & {\cellcolor[rgb]{0.62,0.812,0.878}}M0                                               & {\cellcolor[rgb]{0.62,0.812,0.878}}1                                           & {\cellcolor[rgb]{0.62,0.812,0.878}}                                           & {\cellcolor[rgb]{0.62,0.812,0.878}}                                           & {\cellcolor[rgb]{0.62,0.812,0.878}}1                                           & {\cellcolor[rgb]{0.62,0.812,0.878}}                                           & {\cellcolor[rgb]{0.62,0.812,0.878}}                                            \\ 
\cline{2-8}
                                        & M1                                                                                  & 4.097                                                                          & 1.009--16.644                                                                  & \textcolor{red}{0.049}                                                        & 7.314                                                                          & 1.590--33.631                                                                  & \textcolor{red}{0.011}                                                         \\ 
\hline
\multirow{2}{*}{Clinical Stage}         & {\cellcolor[rgb]{0.62,0.812,0.878}}Stage I+II                                       & {\cellcolor[rgb]{0.62,0.812,0.878}}1                                           & {\cellcolor[rgb]{0.62,0.812,0.878}}                                           & {\cellcolor[rgb]{0.62,0.812,0.878}}                                           & {\cellcolor[rgb]{0.62,0.812,0.878}}1                                           & {\cellcolor[rgb]{0.62,0.812,0.878}}                                           & {\cellcolor[rgb]{0.62,0.812,0.878}}                                            \\ 
\cline{2-8}
                                        & Stage III+IV                                                                        & 1.245                                                                          & 0.882--1.759                                                                   & 0.213                                                                         & 0.621                                                                          & 0.287--1.343                                                                   & 0.226                                                                          \\ 
\hline
\multirow{2}{*}{Surgical Margin status} & {\cellcolor[rgb]{0.62,0.812,0.878}}Negative                                         & {\cellcolor[rgb]{0.62,0.812,0.878}}1                                           & {\cellcolor[rgb]{0.62,0.812,0.878}}                                           & {\cellcolor[rgb]{0.62,0.812,0.878}}                                           & {\cellcolor[rgb]{0.62,0.812,0.878}}1                                           & {\cellcolor[rgb]{0.62,0.812,0.878}}                                           & {\cellcolor[rgb]{0.62,0.812,0.878}}                                            \\ 
\cline{2-8}
                                        & Positive                                                                            & 1.591                                                                          & 1.155--2.191                                                                   & \textcolor{red}{0.004}                                                        & 1.631                                                                          & 1.182--2.250                                                                   & \textcolor{red}{0.003}                                                         \\ 
\hline
\multirow{2}{*}{Tobacco Exposure}       & {\cellcolor[rgb]{0.62,0.812,0.878}}Low                                              & {\cellcolor[rgb]{0.62,0.812,0.878}}1                                           & {\cellcolor[rgb]{0.62,0.812,0.878}}                                           & {\cellcolor[rgb]{0.62,0.812,0.878}}                                           & {\cellcolor[rgb]{0.62,0.812,0.878}}1                                           & {\cellcolor[rgb]{0.62,0.812,0.878}}                                           & {\cellcolor[rgb]{0.62,0.812,0.878}}                                            \\ 
\cline{2-8}
                                        & High                                                                                & 1.364                                                                          & 1.008--1.844                                                                   & \textcolor{red}{0.044}                                                        & 1.363                                                                          & 0.990--1.875                                                                   & 0.058                                                                          \\ 
\hline
\multirow{2}{*}{Gene Expression}                & {\cellcolor[rgb]{0.62,0.812,0.878}}Low                                              & {\cellcolor[rgb]{0.62,0.812,0.878}}1                                           & {\cellcolor[rgb]{0.62,0.812,0.878}}                                           & {\cellcolor[rgb]{0.62,0.812,0.878}}                                           & {\cellcolor[rgb]{0.62,0.812,0.878}}1                                           & {\cellcolor[rgb]{0.62,0.812,0.878}}                                           & {\cellcolor[rgb]{0.62,0.812,0.878}}                                            \\ 
\cline{2-8}
                                        & High                                                                                & 2.101                                                                          & 1.572--2.809                                                                   & \multicolumn{1}{c|}{\textcolor{red}{\textless{} 0.001}}                                     & 2.007                                                                          & 1.490--2.704                                                                   & \multicolumn{1}{c|}{\textcolor{red}{\textless{} 0.001}}                                      \\ 
%\hline
%\multicolumn{8}{|l|}{}                                                                                                                                                                                                                                                                                                                                                                                                                                                                                                                                                                                                           \\ 
%\hline
% \\
\noalign{\hrule height 1.0pt}
\end{tabular}
} % end of \resizebox

\pbox{0.7\columnwidth}{\footnotesize {
(OS: overall survival;
HR: hazard ratio;
CI95\%: 95\% confidence interval;
\protect\textit{p}~value significant code is denoted: red \textless{} 0.05).} %; *** \textless{} 0.001).}                              
}
%\arrayrulecolor{black}
\end{table}



%%%%
\clearpage
\begin{table}[H] 
%\centering
\caption{The top 3 genes with prognostic impacts on HNSCC.}% (ranked by Bonferroni-adjusted \protect\textit{p}~values}

%%%\label{tab:newTable1}
\resizebox{0.9\linewidth}{!}{% \textwidth
\begin{tabular}{|l|l|l|c|c|c|c|c|c|}
\noalign{\hrule height 1.0pt}
\multicolumn{1}{|c|}{} &
  \multicolumn{2}{c|}{} &
  \multicolumn{2}{c|}{\textbf{Kaplan--Meier Survival}} &
  \multicolumn{2}{c|}{\textbf{Cox Univariate}} &
  \multicolumn{2}{c|}{\textbf{Cox Multivariate}} \\ \cline{4-9} 
\multicolumn{1}{|l|}{\multirow{-2}{*}{\textbf{Gene ID}}} &
  \multicolumn{2}{l|}{\multirow{-2}{*}{\textbf{Gene Descriptio}n}} &
  \begin{tabular}[c]{@{}c@{}}\textbf{FDR}\\ \textbf{\protect\textit{p}~Value}\end{tabular} &
  \begin{tabular}[c]{@{}c@{}}\textbf{Bonferroni}\\ \textbf{\protect\textit{p}~Value}\end{tabular} &
  \textbf{HR *} &
  \textbf{CI95\%} &
  \textbf{HR *} &
  \textbf{CI95\%} \\ 
  \hline
CAMK2N1 &
  \multicolumn{2}{l|}{\begin{tabular}[c]{@{}l@{}}calcium/calmodulin-\\ dependent protein\\ kinase II inhibitor 1\end{tabular}} &
  \num{1.63e-5} & %2.9e-7} &
  0.002 &
  2.101 &
  1.572--2.809 &
  2.007 &
  1.490--2.704 \\ \hline
CALML5 &
  \multicolumn{2}{l|} {\acrlong{CALML5}} &
   \num{1.97e-4} & %\num{6.54e-6} & %3.7e-7} &
   0.039 &
   0.51 &
   0.379--0.686 &
   0.493 &
   0.364--0.667 \\ \hline
FCGBP &
  \multicolumn{2}{l|}{\begin{tabular}[c]{@{}l@{}}Fc fragment of\\  IgG binding protein\end{tabular}} &
   \num{4.83e-5} & %1.2e-6} &
   0.008 &
   0.484 &
   0.359--0.653 &
   0.496 &
   0.366--0.674 \\ 
\noalign{\hrule height 1.0pt}
%\multicolumn{9}{|l|}{} \\
%\multicolumn{9}{|l|}%{\multirow{-0}{*}
%{\begin{tabular}[c]{@{}l@{}}
%
%\end{tabular}
%%}\\ % within \multirow
%} \\ % within \multicolumn
%\hline
%\multicolumn{9}{|l|}{ \\ \hline

\end{tabular}%
}
\pbox{0.8\columnwidth}{\footnotesize Selection criteria (fit all): 
(1) Kaplan--Meier Bonferroni-adjusted \protect\textit{p} \textless~0.05; 
(2) Cox's univariate and multivariate HR \textgreater{}= 1.8 or \textless{}= 0.6 in TCGA cohort; 
(3) Cox's univariate and multivariate HR \textgreater{}= 1.8 or \textless{}= 0.6 in GSE65858 cohort.\\
*~Cox's model: \protect\textit{p} \textless 0.001
(HR: hazard ratio; CI95\%: 95\% confidence interval; FDR: \acrlong{fdr}).}

\end{table}
\clearpage


\section{Discussion}


%\subsection*{The Three Biomarkers in Cancer}


%\subsubsection*{The Protein/Pathology Atlas} %still TCGA's RNA-Seq (ok)


%\subsubsection*{Literature Review} % literature (ok)


%\subsection*{Feature Selection for Survival Modeling} % Even there are many $X_1 ... X_n$ physical and social features of patients available for survival modelling in the TCGA.


%\subsection*{The Purpose of Sliding-Window Cutoff Selection}


%\subsection*{Technical Considerations}


\subsection{Limitations of the Study}

\begin{figure}[ht]

\floatbox[{\capbeside\thisfloatsetup{capbesideposition={right,center},capbesidewidth=.25\linewidth,capbesidesep=quad}}]{figure}[\FBwidth]
{%\widefigure
    %\centering
    
    \begin{subfigure}[c]{0.5\textwidth}
    \includegraphics[width=7.5cm]{RplotH2H_TCGA_GSE65858_CoxHR.pdf}
    \caption{Cox's hazard ratios from TCGA HNSCC and GSE65858 (Pearson's correlation coefficient~\cite{Schober2018}, r = 0.27).}
    \end{subfigure}
%\hfill
    \begin{subfigure}[t]{0.15\textwidth}
    \includegraphics[width=2.5cm]{Rplot20_correlation_TCGA_GSE65858_CoxHR.pdf}
    \caption{Correlations of Cox's hazard ratios of those 20 significant genes.}% (Pearson's r = 0.68).}
    \end{subfigure}    
}   
{    \caption{A head-to-head comparison of Cox's hazard ratios from the two datasets (moderate correlation, Pearson's r = 0.68).
%(Pearson's r = 0.68).
%TCGA HNSCC and GSE65858 cohorts were applied for identification and validation of the candidate biomarkers in HNSCC.
%    (\textbf{a}) A total of 5404 genes had Cox's hazard ratios from TCGA HNSCC and GSE65858
%(Pearson's correlation, r = 0.27).
    %\acrshort{fdr}-\protect\textit{p}~values of less than 0.05 in TCGA HNSCC.
%    \textcolor{red}{CAMK2N1}, \textcolor{red}{CALML5}, \textcolor{red}{FCGBP}, and 17 other genes (marked in \textcolor{black}{black}) had hazard ratios (HRs) >$1.8$ or <$0.6$.
%    \textcolor{red}{Red spots}: $HR$s > 1.0 in TCGA HNSCC.
%    \textcolor{green}{Green spots}: $HR$s < 1.0 in TCGA HNSCC.
%    Sizes of spots: bigger for Kaplan--Meier \protect\textit{p}~values in TCGA HNSCC. %Please ensure intended meaning is retained.
%    (\textbf{b}) The 20 genes were extracted and shown. The hazard ratios of those genes have a moderate correlation between the two cohorts
%(Pearson's r = 0.68).
%    (X-axis: Hazard ratios of Cox proportional hazard regression model from TCGA HNSCC;
%    y-axis: Those values from GSE65858; TCGA: \acrlong{tcga}; HNSCC: \acrlong{hnscc}.)
    }}
%%%\label{fig:hazards_head2head_TCGA_GSE65858}
\end{figure}


%\subsection{Future Directions in Translational Medicine}


%\subsubsection*{Proteomics Validation}


%\subsubsection*{Laboratory Validation}


%\subsubsection*{Cancer Type-Agnostic Study}


\subsection{Holistic Cancer Care} 

\begin{figure}[ht]

\begin{minipage}[c]{0.60\linewidth}
%\floatbox[{\capbeside\thisfloatsetup{capbesideposition={right,center},capbesidewidth=.25\linewidth,capbesidesep=quad}}]{figure}[\FBwidth]
%\centering

\setlength{\unitlength}{.78cm}
\begin{picture}(20, 10)(0,0) %(1,0.55038404)%
%\centering
  \put(0,0){\includegraphics[width=9.0cm]{Figure_5_holisticCare.pdf}}%
  \put(0.4, 9.3){\fontfamily{qcr}\selectfont
  \textbf{[Physician]}}%
  \put(6.0, 7.8){\fontfamily{qcr}\selectfont
  HNSCC}
\end{picture}
%\includegraphics[width=10cm]{Figure_5_holisticCare.pdf}
\end{minipage}
\hfill
\begin{minipage}[c]{0.35\linewidth}
%{\caption{
%The concept of holistic care for \acrshort{hnscc} patients. %MDPI: please confirm if the extra lines and frame need to be deleted.

Beyond carcinogenesis\\
the mind--brain--body axis~\cite{Hsiao2012}

\begin{outline}
\1 stress %ful environment(giant \textcolor{black}{black} arrow) 
will trigger an emotional response
\1 brain releases stress hormones and inflammation signals% in response to negative emotions.
\1 %The  body's internal environment (cells) 
altering epigenetic control in gene regulation and mRNA expression of cells
%Over a long time, the tissue/cells will be transformed into dysplasia and then 
\1 carcinogenesis~\cite{Lutgendorf2010,Powell2013,Iftikhar2021} with help from known carcinogens
\end{outline}

\end{minipage}

%%%\label{fig:figure5}
\end{figure}
\clearpage



\begin{figure}[ht]

\begin{minipage}[c]{0.50\linewidth}
%\floatbox[{\capbeside\thisfloatsetup{capbesideposition={right,center},capbesidewidth=.25\linewidth,capbesidesep=quad}}]{figure}[\FBwidth]
%\centering

\setlength{\unitlength}{.78cm}
\begin{picture}(20, 10)(0,0) %(1,0.55038404)%
%\centering
  \put(0,0){\includegraphics[width=9.0cm]{Figure_5_holisticCare.pdf}}%
  \put(0.4, 9.3){\fontfamily{qcr}\selectfont
  \textbf{[Physician]}}%
  \put(6.0, 7.8){\fontfamily{qcr}\selectfont
  HNSCC}
\end{picture}
%\includegraphics[width=10cm]{Figure_5_holisticCare.pdf}
\end{minipage}
\hfill
\begin{minipage}[c]{0.45\linewidth}
Holistic cancer care~\cite{Mehta2019,Iftikhar2021}:  
\begin{outline}
\1 to support cancer patients' spiritual, emotional, physical, and socioeconomic needs
\1 to give the physical care: medication therapy or surgery. 
\1 therapeutic relationship (TR)~\cite{Rogers1979}, the physicians' spiritual properties (empathy, sympathy, and compassion) will engage cancer patients
\1 to induce their self-compassion to gain resilience against the disease through their mind--brain--body axis~\cite{Hsiao2012}
%Thus, we suggest that electric healthcare records (EHR) should include physical, pathological, and psychological data, and even more spiritual information. The \acrshort{tcga} might collect those "holistic features"  (\textcolor{green}{green} dashed line) for further study of personalized medicine.
\end{outline}

\end{minipage}

%%%\label{fig:figure5}
\end{figure}
\clearpage




\section*{Conclusions} % 5
\begin{outline}
\1 Three biomarker candidates---CAMK2N1, CALML5, and FCGBP---which are all heavily associated with the prognosis of \acrlong{os}.
\1 The microenvironment of HNSCC, influenced by the mind--brain--body axis~\cite{Hsiao2012}
    \2 further exploration and understanding using holistic multi-parametric approaches
    \2 the \acrshort{tcga} must collect those "holistic features" for further study of personalized medicine
\1 Good using placebo effect
    \2 the confidence might promote healing through a mind--brain--body connection manner
\1 Mindfulness meditation is helpful to cancer patients
    \2 confess for not taking care of their bodies and spirits in the past
    \2 give sincere thanks for their physical body's hard work
\end{outline}

%人俯仰於天地之間
%順從四季氣候變化
%保養正氣陶冶性情
%自我療癒身心靈疾病
%自他不二 心存正念 向內看,解決苦之源
%1.衷心懺悔
%2.真心感謝
%3.誠意祝福
%4.永存善念 慈悲
%5.心無恐懼: 情志養生
%每位醫師都可以成為「創傷知情者」幫助我們身邊的病患,懂他的心靈創傷,讓他有安全感, 才有機會改變疾病的走向
%"每一位患者都有自癒能力, 我們知情之後, 也要逐步讓他本人知情, 看見之後, 在良好的「治療關係」中, 協助他們漸漸找回自己的療癒。(以他們自己的腳步)"
%解說:患者的自癒力就是復原力(resilience),  強調「治療關係」(安全、 信任、 分享權力、 自決) 以及”知情”(暸解過往創傷經驗對自身的影響,進而開始療癒的過程)的重要性。
%%%%%%%%%%%%%%%%%%%%%%%%%%%%%%%%%%%%%%%%%%%%%%%%%%%%%%%%%%%%%%%
\clearpage
%%%%%%%
\thispagestyle{empty} % No slide header and footer

\bibliographystyle{unsrt}
\bibliography{TCGA_margin_cutoff.bib}

\clearpage

%------------------------------------------------

\thispagestyle{empty} % No slide header and footer

\begin{tikzpicture}[remember picture,overlay] % Background box
  \node [xshift=\paperwidth/2,yshift=\paperheight/2] at (current page.south west)
    [rectangle,fill,inner sep=0pt,minimum width=\paperwidth,minimum height=\paperheight/2.1,top color=myblue,bottom color=myblue]{}; % Change the height of the box, its colors and position on the page here
\end{tikzpicture}
% Text within the box
\begin{flushright}
  \vspace{1.6cm}
  \color{white}\sffamily
  {\bfseries\Large Comments and Suggestions\par}% Title
  \vspace{0.5cm}
  \normalsize
%  \myauthor\par % Author name
%  \mydate\par % Date
  \myuni\par
  \vfill
\end{flushright}

%----------------------------------------------------------------------------------------
%
\end{document}


%%%%
% spared code

%------------------------------------------------


\section*{Verbatim}

How to include a theorem in this presentation:
\begin{verbatim}
\mybox{0.8\textwidth}{
\begin{theorem}[Murphy (1949)]
Anything that can go wrong, will go wrong.
\end{theorem}
}
\end{verbatim}

\clearpage

%------------------------------------------------



\subsection*{Figure}

%\clearfloatsetup{figure}
%\floatsetup[figure]{style=no,capposition=beside,capbesideposition={center,inside},capbesideframe=yes,facing=yes}

\begin{figure}[ht]
\floatbox[{\capbeside\thisfloatsetup{capbesideposition={right,center},capbesidewidth=.35\linewidth,capbesidesep=quad}}]{figure}[\FBwidth]
{\caption{\texttt{capbesidesep=quad}. I want to thank Sonia. The caption can contain any text but needs to describe the image with enough detail for a reader to completely understand the image.}}
{\includegraphics[width=0.6\textwidth]{placeholder.jpg}}
\end{figure}
%\includegraphics[width=9cm]{placeholder}

%\sidecaptionvpos{c}
%\caption{I want to thank Sonia.}
%\end{captionbeside}


\clearpage

%------------------------------------------------

